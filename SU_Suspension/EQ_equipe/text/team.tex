
 
 \newpage \section*{Sébastien GIROIX} 
 \par Priorité: avancement de la direction. Devis crémaillère
\paragraph{$\bigstar$ devis steering rack} crémaillère Narrco (environ 500 eur ttc). Dèmarche nouveau fournisseur à démander à Romain Martin. Ne pas avancer sur la maquette si le devis n'as pas été envoyé. Utiliser le modèle de devis en anglais
\paragraph{choix visserie steering SU\_A0900} garder la meme que optimus
\paragraph{lire tutoriel BOM} (avec Matthieu CORNU, Enzo POTTEZ, Rémi FERRAND, Pierre-Emmanuel ARIAUX, Robin NIERMARECHAL, Raphaël CHARLET, Adèle LEFEVRE) des questions?
\paragraph{design steering shaft ST\_A0200} crèation de la maquette et structure du git. joint de cardan simple.Cinématique régler collisions. Conception liasion pivot de la colonne avec le chassis (verifiier solution de Optimus). Liaison colonne - crémaillère: quel type de joint ? On va utiliser une solution acier optimisé en masse pour garder la fiabilité des soudures.
\paragraph{jalons scolarité} (avec Matthieu CORNU, Michele SCHIO, Enzo POTTEZ, Rémi FERRAND, Pierre-Emmanuel ARIAUX, Robin NIERMARECHAL, Raphaël CHARLET, Adèle LEFEVRE) lister ses propres dates des examens ou des soutenances pour une maj du gantt
\paragraph{lire tutoriel Devis} (avec Matthieu CORNU, Enzo POTTEZ, Rémi FERRAND, Pierre-Emmanuel ARIAUX, Robin NIERMARECHAL, Raphaël CHARLET, Adèle LEFEVRE) des questions?
\paragraph{devis steering shaft ST\_A0200} quels joints entre la colonne et la crémaillère ? est ce qu'il y a qqch à commander ?
\paragraph{design steering rack cover ST\_A0300} 
\paragraph{taille rotules steering SU\_A0900} cas de charge MKI Vs catalogue fournisseur

 
 \newpage \section*{Rémi FERRAND} 
 \par Tu es le nouveau chargé des disques de frein (suite du travail d'Enzo POTTEZ)
\paragraph{choix brake disc front} trouvel l'équivalent disque plein de Optimus (disque avec frette). Trouver un disque persé qui soit compatible avec le discque plein. Il faout que le tout soit intégrable sur les moyeu de Vultanix pour la phase de test
\paragraph{lire tutoriel BOM} (avec Matthieu CORNU, Enzo POTTEZ, Pierre-Emmanuel ARIAUX, Sébastien GIROIX, Robin NIERMARECHAL, Raphaël CHARLET, Adèle LEFEVRE) des questions?
\paragraph{étude disque de frein perçés} 1- gain de masse eur/gram gagné
2- compatibilité disques entre plein et perçé et Vulcanix
3 - Approvisionnement disques: fournisseurs ? devis ?
4- definir un protocole d'essai avec Vulcanix

\paragraph{placement brake caliper front} front upright. definir la position de l'étrier par rapport au disque
\paragraph{lire tutoriel Devis} (avec Matthieu CORNU, Enzo POTTEZ, Pierre-Emmanuel ARIAUX, Sébastien GIROIX, Robin NIERMARECHAL, Raphaël CHARLET, Adèle LEFEVRE) des questions?
\paragraph{placement brake caliper rear} rear upright. definir la position de l'étrier par rapport au disque
\paragraph{choix brake disc rear} trouvel l'équivalent disque plein de Optimus (disque avec frette). Trouver un disque persé qui soit compatible avec le discque plein. Il faout que le tout soit intégrable sur les moyeu de Vultanix pour la phase de test
\paragraph{jalons scolarité} (avec Matthieu CORNU, Michele SCHIO, Enzo POTTEZ, Pierre-Emmanuel ARIAUX, Sébastien GIROIX, Robin NIERMARECHAL, Raphaël CHARLET, Adèle LEFEVRE) lister ses propres dates des examens ou des soutenances pour une maj du gantt
\paragraph{planning campagne d'essai} avec vulcanix.

 
 \newpage \section*{Thibaud LASSUS} 
 \par coucou
\paragraph{planning reu tech nAs} 
\paragraph{rapport techniques Simon Laurent} 
\paragraph{planning reu tech Janolin Houx A} 
\paragraph{reu tech roue équipée} monoh, reedom, jacques, 
\paragraph{pièce titane} (avec Martin KWCY) choisir une pièce titane non critique pour la fabrication en additif

 
 \newpage \section*{Pierre-Emmanuel ARIAUX} 
 \par tu travaillera sur la BAR avec Robin, la maquette à besoin d'avancer. MSO pour le positionnement (avec le chassis)
\paragraph{design rear ARB SU\_A15} barre de torsion + liaison avec le chassis (palier laiton ?). tube acier Optimus CD45 diamètre ext 15 mm ( La Gir)
\paragraph{lire tutoriel BOM} (avec Matthieu CORNU, Enzo POTTEZ, Rémi FERRAND, Sébastien GIROIX, Robin NIERMARECHAL, Raphaël CHARLET, Adèle LEFEVRE) des questions?
\paragraph{simulation torsion bar SU\_A1500} 
\paragraph{lire tutoriel Devis} (avec Matthieu CORNU, Enzo POTTEZ, Rémi FERRAND, Sébastien GIROIX, Robin NIERMARECHAL, Raphaël CHARLET, Adèle LEFEVRE) des questions?
\paragraph{simulation torsion bar SU\_A1400} 
\paragraph{design front ARB SU\_A14} barre de torsion + liaison avec le chassis (palier laiton ?). tube acier Optimus CD45 diamètre ext 15 mm ( La Gir)
\paragraph{jalons scolarité} (avec Matthieu CORNU, Michele SCHIO, Enzo POTTEZ, Rémi FERRAND, Sébastien GIROIX, Robin NIERMARECHAL, Raphaël CHARLET, Adèle LEFEVRE) lister ses propres dates des examens ou des soutenances pour une maj du gantt

 
 \newpage \section*{Martin KWCY} 
 \par coucou
\paragraph{placement crémaillère et biellette direction} placer les points Lotus
\paragraph{pièce titane} (avec Thibaud LASSUS) choisir une pièce titane non critique pour la fabrication en additif
\paragraph{cas de charge et coeff de secu} nominale limite ultime ? coeff de secu ?

 
 \newpage \section*{Robin NIERMARECHAL} 
 \par tu es le nouveau chargé de la conception des couteaux de la barre anti roulis. tu travaillera avec Pierre-Emmanuel
\paragraph{lire tutoriel BOM} (avec Matthieu CORNU, Enzo POTTEZ, Rémi FERRAND, Pierre-Emmanuel ARIAUX, Sébastien GIROIX, Raphaël CHARLET, Adèle LEFEVRE) des questions?
\paragraph{simulation arb blades SU\_A1400} 
\paragraph{lire tutoriel Devis} (avec Matthieu CORNU, Enzo POTTEZ, Rémi FERRAND, Pierre-Emmanuel ARIAUX, Sébastien GIROIX, Raphaël CHARLET, Adèle LEFEVRE) des questions?
\paragraph{design rear ARB SU\_A15} couteaux tournants et support de fixation des coutaux à la barre de torsion. Biellette de liaison avec le basculeurs ( assemblage avec rotules)
\paragraph{simulation arb blades SU\_A1500} 
\paragraph{design front ARB SU\_A14} couteaux tournants et support de fixation des coutaux à la barre de torsion. Biellette de liaison avec le basculeurs ( assemblage avec rotules)
\paragraph{rotules ARB front et rear} 
\paragraph{jalons scolarité} (avec Matthieu CORNU, Michele SCHIO, Enzo POTTEZ, Rémi FERRAND, Pierre-Emmanuel ARIAUX, Sébastien GIROIX, Raphaël CHARLET, Adèle LEFEVRE) lister ses propres dates des examens ou des soutenances pour une maj du gantt

 
 \newpage \section*{Michele SCHIO} 
 \par devis romain priorité
\paragraph{placement rear ARB chassis} utiliser le tube moteur pour se fixer avec la barre de torsion
\paragraph{placement front ARB chassis} 
\paragraph{determiner efforts BAR} en utilisant le ARB ratio pour le déplacement
\paragraph{simulation rear upright SU\_A1100} cas de charge MKI
\paragraph{rediger tutoriel Simulation FEA} discuter de la methode avec Calixte
\paragraph{simulation front hub WT\_A0200} cas de charge MKI

\paragraph{design rear upright SU\_A1100} meme structure d'optimus ( enlever le parallelogramme) Placement de l'étrier. Placement de la toe rod.

\paragraph{jalons scolarité} (avec Matthieu CORNU, Enzo POTTEZ, Rémi FERRAND, Pierre-Emmanuel ARIAUX, Sébastien GIROIX, Robin NIERMARECHAL, Raphaël CHARLET, Adèle LEFEVRE) lister ses propres dates des examens ou des soutenances pour une maj du gantt
\paragraph{design front upright SU\_A1000} definir la position de létrier et des biellettes de direction 
\paragraph{rediger tutoriel Vues éclatéess} 
\paragraph{choisir fournisseur} skf, getecno ou askuball ?? mettre les catalogues à disposition de l'équipe pour choisir les modèles et faire les devis. rentrer les rotules dans le fichier de gestion LAS . Askubal est mieux car ils proposent déjà une selection de prodiits spécifique pour le FS. 
\paragraph{$\bigstar$ envoyer devis à Romain} deux budgets : 1 centrale et 1 Bron. Tu doit leur avoir envoyé toutes tes commandes le 15 novembre (considère le 8 novembre pour avoir de la marge).
Centrale : 2,8k€ à utiliser
Bron : 3k€ à utiliser
Pour les deux, les commandes doivent être payées avant les vacances de noël, cad avant le 21 décembre 2019.
Cad :
- soit payées en avance, dans ce cas c'est bon
- soit livrées avant le 21 décembre 2019.
\paragraph{placement toe rod rear} 
\paragraph{collision montage visserie sur le chassis} macro visserie Brice
\paragraph{OptimumG dynamics no report} repondre à Rouelle en lui disant que on a pas pu utiliser son logiciel (la licence à été activée ??)
\paragraph{$\bigstar$ devis pneus compet} 9 pneus slick + 4 pneus wet (environ 2400 eur ttc). nouveau fournisseur
\paragraph{simulation front upright SU\_A1000} cas de charge MKI
\paragraph{determiner raideur BAR} modèle et calculs de MKI python
\paragraph{shock choice} determiner la raideur des ressorts (parmi celles à disposition). Normalement c'est déjà fait par MKI --> fichiers de calcul dans le git Ressources2020 
\paragraph{choix roulements WT\_A02 A\_03} On ne change pas les roulements, on ne refait pas le calcul de vies des roulements. Les soucis d'optimus étaient liés à un mauvais montage ou à une manque de precharge. On va commander un douille speciale pour les écrous à encoche pour Invictus de telle façon à maitriser la précharge (demarche classique par essai). Démander à NGO la ref de la commande.
\paragraph{design rear hub WT\_A0300} gain de perfo goujon Ti (> 1 eur/gram gagné ?). pièce de FsaeParts.com. Vérifier reglement pour les goujons. Fournisseur + devis ?
\paragraph{simulation rear hub WT\_0300} cas de charge MKI
\paragraph{choix k-nut tulipe} 
\paragraph{$\bigstar$ devis jantes} 2 jantes Oz Mg R13 chez Reverchon environ /%= eur ttc. valider le nombre de jante avec MKI et TLS
\paragraph{$\bigstar$ devis wheel bearing}  6 roulements (2 extra) en utilisant les 4 qui sont dejà à Bron. envoron 1500 eur ttc. 
\paragraph{rediger tutorier Devis} 
\paragraph{iterations cellule arrière} message Calixthe slack: reculer de 20mm en Y le Front Upper A Arm Point  des Rear A-Arms,
\paragraph{entretoise}  a standardisaer avec celles du A-Arms: utiliser la taille imposée par le chassis (20 mm). Démander à Calixthe la nouvelle distance des chapes chassis 
\paragraph{choix visserie front hub WT\_A0200} lock nut + washer 
\paragraph{rediger tutoriel Contraintes} 
\paragraph{design fron hub WT\_A0200} 
\paragraph{ecrou tulipe} nilstop verifier ? aeronut (UK) conv TLS slack --> 23 filet nilstop. Reglement 2 fillets dépassants --< longueur de l'écrou

 
 \newpage \section*{Raphaël CHARLET} 
 \par personal notes
\paragraph{lire tutoriel BOM} (avec Matthieu CORNU, Enzo POTTEZ, Rémi FERRAND, Pierre-Emmanuel ARIAUX, Sébastien GIROIX, Robin NIERMARECHAL, Adèle LEFEVRE) des questions?
\paragraph{design rod-end support SU\_A0900} threaded rod-end. Se reinsegner avec la commande des rotules. Collage alu - carbone comme dans le triangles
\paragraph{taille rotules pullbar SU\_A12 A13} cas de charge MKI Vs catalogue fournisseur
\paragraph{lire tutoriel Devis} (avec Matthieu CORNU, Enzo POTTEZ, Rémi FERRAND, Pierre-Emmanuel ARIAUX, Sébastien GIROIX, Robin NIERMARECHAL, Adèle LEFEVRE) des questions?
\paragraph{design rod-end SU\_A12 A\_13} 
\paragraph{simulation inerts SU\_A0100} simulation EF des inserts avec les cas de chage Mécamaster
\paragraph{choix visserie triangles frontSU\_A01 A\_02 A\_03 A\_04} garder la meme que Optimus
\paragraph{design rod-end ST\_A0400} 
\paragraph{jalons scolarité} (avec Matthieu CORNU, Michele SCHIO, Enzo POTTEZ, Rémi FERRAND, Pierre-Emmanuel ARIAUX, Sébastien GIROIX, Robin NIERMARECHAL, Adèle LEFEVRE) lister ses propres dates des examens ou des soutenances pour une maj du gantt
\paragraph{taille rotules triangles frontSU\_A01 A\_02 A\_03 A\_04} cas de charge MKI Vs catalogue fournisseur. Choisir le pir de cas dans la fauille des calcul des cas de charge. Coeff de sécurité sur la charge max d'au moins 2
\paragraph{choix visserie pullbar SU\_A12 A13} garder les visses d'optimus (cf reglement)
\paragraph{design rod-end support} buttée mécanique pour les inserts porte-moyeu qui reprennent la suspension ( avec circlip). Du coté chassis les inserts n'ont pas besion des circlips, la butée mécanique (cf Optimus) est souffisante. Les rotules sont montées serrées sur les inserts ( cf ajoustement mécanique Optimus). Aurelien Binner est disponible pour toutes questions.

 
 \newpage \section*{Enzo POTTEZ} 
 \par tu travaillera le collage des triangles (Raphael s'occupera de la conception méca) et le freinage (Rémi s'occupera d'étude des disques )
\paragraph{lire tutoriel BOM} (avec Matthieu CORNU, Rémi FERRAND, Pierre-Emmanuel ARIAUX, Sébastien GIROIX, Robin NIERMARECHAL, Raphaël CHARLET, Adèle LEFEVRE) des questions?
\paragraph{moidèle de freinage} verifier les calculs et les hypothèses du modèle de freinage. présenter ce qui à été fait aux nAs
\paragraph{lire tutoriel Devis} (avec Matthieu CORNU, Rémi FERRAND, Pierre-Emmanuel ARIAUX, Sébastien GIROIX, Robin NIERMARECHAL, Raphaël CHARLET, Adèle LEFEVRE) des questions?
\paragraph{$\bigstar$ verifier avec le fournisseur la colle} collage alu - carbone. Optimus avait utilisé la colle DP490 de 3M. Demander des conseils directement à 3M en leur explicant le notre application. est ce que la temperature est importante ?
\paragraph{jalons scolarité} (avec Matthieu CORNU, Michele SCHIO, Rémi FERRAND, Pierre-Emmanuel ARIAUX, Sébastien GIROIX, Robin NIERMARECHAL, Raphaël CHARLET, Adèle LEFEVRE) lister ses propres dates des examens ou des soutenances pour une maj du gantt
\paragraph{$\bigstar$ devis tubes essai triangles} fournisseur: Mateduc composites (environ 500 eur ttc avec decoupe). Estimer d'abord la longueur nécessaire à effectuer 20 essay sur des tubes de 20 cm. grandir la surface de collage (et donc le diamètre des tubes) en suivant la démarche du protocole d'essai
\paragraph{essai rupture collage triangles} essai en rupture pour comparaison avec la méthode du passée ( Atomix, Vulcanix et Optimus). Essai en fatigue: à vérifier si c'est faisable. L'essai en faitigue pour envoron 1000 cycles permettrait de voir l'évolution de la charge de rupture avec 1000 cycles. Les essais seront faits au laboratoir LTDS de l'école. Démander à Aurelien Bienner la ref du prof à contacter
\paragraph{$\bigstar$ revoir process de collage triangles} donner un protocole de collage des triangles en adaptant les méthodes d'Atomix, Vulcanix et Optimus. 
1- Sillons dans la surfca alu des cylindres ( avec la vitesse d'avance de l'outi de coupe) 
2- Sablage des pièce alus pour mieux nettoyer
3- Ponçage tubes carbone intérieur (quel papirer à ponser?)
4- Nettoyage pièces ( 3 poduits - voir Atomix) + séchage
5- Ajout de postilles rond en plastique pour empecher à la colle de déscendre ( Atomix)
6- Ajout colle sur les inserts + enfoncer en tournant
7- Laisser secher 7 jours à temp ambiente
RQ - Augmenter la surface de collage en passant de 30 mm (Optimus) à 35 mm et en augmentant le diamètre des tubes de carbone. Il faut obtenir 1.5 - 2 fois la surface de collage d'optimus avec ces deux modif.
RQ - Pour les essais réutiliser les rotules d'Optimus ( reprendre les inserts d'optimus)
RQ - pour les insert du coté porte-moyeu on usine directement le cylindre sur l'insert pour augmenter la fiabilité (le décollages d'Optimus étaient à cet endroit là)
RQ - Aurelien Bienner (Reedom) a donné sa disponibilité pour toutes questions

 
 \newpage \section*{Matthieu CORNU} 
 \par tu es le nouveau respo de l'assemblage basculeurs et amortisseurs 
\paragraph{lire tutoriel BOM} (avec Enzo POTTEZ, Rémi FERRAND, Pierre-Emmanuel ARIAUX, Sébastien GIROIX, Robin NIERMARECHAL, Raphaël CHARLET, Adèle LEFEVRE) des questions?
\paragraph{état ressorts Optimus} quels ressorts sont disponibles d'après le crash d'Optimus ?
\paragraph{BOM rocker assy SU\_A0600} determiner le type de roulements pour la liaison avec le chassis. les rotules seront comandée en lien avec les triangles et les pull bars. Choisir la visserie en s'inspirant d'Optimus
\paragraph{sourveiller les chapes des basculeurs} elles serond déssinées par le departement du chassis. Suivre l'avancement des chapes afin de bien les intégrer dans l'assemblage des basculeurs
\paragraph{jalons scolarité} (avec Michele SCHIO, Enzo POTTEZ, Rémi FERRAND, Pierre-Emmanuel ARIAUX, Sébastien GIROIX, Robin NIERMARECHAL, Raphaël CHARLET, Adèle LEFEVRE) lister ses propres dates des examens ou des soutenances pour une maj du gantt
\paragraph{roulements rockers front + rear} 
\paragraph{lire tutoriel Devis} (avec Enzo POTTEZ, Rémi FERRAND, Pierre-Emmanuel ARIAUX, Sébastien GIROIX, Robin NIERMARECHAL, Raphaël CHARLET, Adèle LEFEVRE) des questions?
\paragraph{simulation rockers SU\_A0600} cas de charge MKI
\paragraph{choix visserie rockers SU\_A06 A08} gardere la meme que optimus

\paragraph{$\bigstar$ devis kit reparation Ohlins} se reinsegner sur le kit de réparation des amortisseurs Ohlink Mk II. Combien faut-il en acheter pour reparer 4 (+1 vulcanix) amortisseurs d'Optimus? Quel fournisseur ? Effectuer le devis 
\paragraph{design rocker SU\_A0600 SU\_A0800} proposer un assemblage pour la liaison pivot en suivant la solution utlisée par Jacques dans le pédalier (butée à rouleaux + bague en laiton + vis épaulée + rondelle) Voir photo. Choisir les composants et les rentrer dans la maquette en modifiant la nomenclature si necessaire.

 
 \newpage \section*{Adèle LEFEVRE} 
 \par tu travaillerà principalement sur les frette de frein (avant et arrière). Rémi choisir° les disques et Michele s'occupera du moyeu
\paragraph{jalons scolarité} (avec Matthieu CORNU, Michele SCHIO, Enzo POTTEZ, Rémi FERRAND, Pierre-Emmanuel ARIAUX, Sébastien GIROIX, Robin NIERMARECHAL, Raphaël CHARLET) lister ses propres dates des examens ou des soutenances pour une maj du gantt
\paragraph{lire tutoriel BOM} (avec Matthieu CORNU, Enzo POTTEZ, Rémi FERRAND, Pierre-Emmanuel ARIAUX, Sébastien GIROIX, Robin NIERMARECHAL, Raphaël CHARLET) des questions?
\paragraph{design frette de frein arrière} maquette et structure du git. s'inspirer d'optimus
\paragraph{design frette de frein avant} maquette catia et structure du git
\paragraph{lire tutoriel Devis} (avec Matthieu CORNU, Enzo POTTEZ, Rémi FERRAND, Pierre-Emmanuel ARIAUX, Sébastien GIROIX, Robin NIERMARECHAL, Raphaël CHARLET) des questions?
