\section{Tutoriel Devis}

\subsection{Accéder au fichier de Synthèse}
\begin{enumerate}
	\item Se connecter au   \href{https://docs.google.com/spreadsheets/d/1zmN-TiNXafv8fsJI3vR6wi9P5roOi5zo0ymeUQcS7vQ/edit?usp=sharing}{Gsheet de gestion LAS} 
	
	\item Rentrer dans l’onglet “Synthèse” et utiliser le filtre dans le colonnes F et G\\
	\rfigure{15cm}{img/devis1.png}{Onglet synthèse du fichier de gestion LAS}
	
	\item Cliquer sur l'icône du filtre dans la cellule F1 (make/buy/optimus), puis “Effacer”, puis “b” pour sélectionner les pièces à acheter
	\item Cliquer sur l’icône du filtre dans la cellule G1 (respo), puis “Effacer”, puis sélectionner votre trigramme
	
\end{enumerate}

\subsection{Enregistrer l'avancement d'un devis}
\begin{enumerate}
	\item Indiquer “m”, “b” ou “o” sur votres pièces (optimus signifie que la pièce a été récupérée de la voiture du 2019)
	\item Saisir la quantité de chaque pièce à acheter
	\item Repérer le fournisseur \footnote{Utiliser l’onglet “Budget” pour accéder au devis de l’année 2019. Les liens des documents PDF sont disponibles et pointent vers epsabox} dans l’onglet Fournisseurs et indiquer son nom dans la colonne 	“supplier” de l’onglet Synthèse en faisant le lien vers l’onglet Fournisseurs. Exemple​ : pour insérer le fournisseur ARRK LCO dans la colonne Supplier de Synthèse il faut écrire \texttt{=​ Fournisseurs!A11}
	\item Envoyer une mail au fournisseur \footnote{Tu trouveras le contacte du fournisseur dans les devis de l’année dernier (onglet Budget) ou en recherchant sur internet. Si t’as du mal, poste un petit message sur slack} en mettant en copie et en suivant les modèles (proposés
	plus en bas) pour demander un devis PDF des pièces.
	\item Revenir dans l’onglet Synthèse et mettre “-” dans la bonne cellule de la colonne “devis” (elle deviendra jaune)
\	\item Envoyer le devis PDF à Romain Martin (​\textit{ romain.matin@ecl18.ec-lyon.fr}​ ) en me mettant en copie (​ \textit{michele.schio@ecl18.ec-lyon.fr}​ ) puis revenir dans l’onglet Synthèse et mettre “x” dans la bonne cellule de la colonne “devis” (elle deviendra verte)
	
\end{enumerate}

\newpage
\subsection{Modèle devis français}
\textbf{OB: Écurie Piston Sport Auto - Devis Formula Student}

Bonjour,\\

\par je suis un élève ingénieur de l’École Centrale de Lyon (ECL) et membre du département Liaison au Sol à l’Écurie Piston Sport Auto (EPSA), l’association à travers laquelle l’ECL participe chaque
année à la compétition du Formula Student.

\par Je vous contacte pour vous demander un devis (en format PDF) pour les pièces suivantes :
\begin{itemize}
	\item REFpièceA quantité: 3
	\item 
\end{itemize}




\\[1cm] Adresse de facturation:\\
Ecole Centrale de Lyon\\
Service Facturation Bat Z2\\
36 avenue Guy de Collongue 69134 ECULLY, France\\

\par Je reste à disposition pour toute information supplémentaire. Je vous remercie par avance de votre réponse rapide.


\par Bien cordialement\\
SIGNATURE AVEC NUMÉRO DE PORTABLE


\subsection{Modèle devis anglais}

\textbf{OB: Écurie Piston Sport Auto - Formula Student Quotation}

Dear Sir or Madam,\\

\par I am an engineering student from École Centrale de Lyon (France) at the "Ecurie Piston Sport
Auto" Formula Student team.
\par I am writing to you in order to ask for a quotation (PDF format if possible) for the following items:
\begin{itemize}
	\item REFpièceA quantity: 3
	\item 
\end{itemize}

\par You can use the following contact information if needed.\\
\textbf{Billing Address:}\\
Ecole Centrale de Lyon\\
Service Facturation Bat Z2\\
36 avenue Guy de Collongue 69134 ECULLY, France\\[.5cm]
\textbf{Delivery Address:}\\
Ecole Centrale de Lyon\\
MS-GM-GC Bat H10 Béatrice Chervet\\
36 avenue Guy de Collongue 69134 ECULLY, France\\
\par I am looking forward to hearing your reply and to answer any further question.
\par Yours faithfully,\\
SIGNATURE AVEC NUMÉRO DE PORTABLE