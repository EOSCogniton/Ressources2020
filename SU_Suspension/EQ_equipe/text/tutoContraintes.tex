\section{tutoriel Contraintes des assemblages Catia}

\par Afin de bien insérer toutes contraintes dans les assemblages Catia il est impératif des les renommer et de les ordonner en dossiers dans l'arbre des contraintes. Voici la procédure:

\rfigure{5cm}{img/contr1.jpg}{Créer une nouvelle contrainte et la sélectionner}
\rfigure{6cm}{img/contr2.jpg}{click droit puis rentrer dans les propriétés}
\rfigure{6cm}{img/contr3.jpg}{dans l'onglet "Contrainte" renommer proprement }\\

\rfigure{5cm}{img/contr4.jpg}{Pour réunir plusieur contraintes dans un dossier les sélectionner d'abord avec la touche \texttt{Ctrl}}
\rfigure{10cm}{img/contr5.jpg}{click droit puis "Group in a new set" (\texttt{Ctrl+W})}\\

\rfigure{5cm}{img/contr6.jpg}{pour renommer le dossier de contraintes, click droit puis propriétés}
\rfigure{7cm}{img/contr7.jpg}{dans l'onglet "Feature Properties" nommer proprement}
\rfigure{3cm}{img/contr8.jpg}{n'oublie pas de mettre à jour l'assemblage après la création de nouvelles contraintes}