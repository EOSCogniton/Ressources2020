\section{Tutoriel BOM}

\par Afin de bien réaliser la table des matière (en anglais Bill Of Materials) il est nécessaire de saisier correctement des données des pièces dans la structure arborescente de Catia. \textcolor{green}{Il est impératif que le nom du fichiers dans le git soit du type WT\textunderscore A0100 (description) en suivant la nomenclature du COST et en saisissant les mêmes données dans le propriétés de Catia.}

\rfigure{5cm}{img/prop1.jpg}{selectionner le produit}
\rfigure{5cm}{img/prop2.jpg}{rentrer dans la pièce}
\rfigure{6cm}{img/prop3.jpg}{click droit, puis propriétés}\\
\rfigure{10cm}{img/prop4.jpg}{1: nom de la pièce à l'intérieur de l'assemblage (c'est juste pour se repérer, cette case n'est disponible que pour les assemblages-produit) 2 et 3: nomenclature du COST; 4: fabriqué/ acheté; 5: remarques (facultatif)}

